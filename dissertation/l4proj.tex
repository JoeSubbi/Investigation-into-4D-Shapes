% REMEMBER: You must not plagiarise anything in your report. Be extremely careful.

\documentclass{l4proj}

% put any additional packages here
\usepackage{amsmath}

\begin{document}

%==============================================================================
%% METADATA
\title{Level 4 Project: An Investigation of How Best to Extend a Cross Section of 4 Dimensional Objects}
\author{Joe Subbiani}
\date{March 18, 2022}

\maketitle

%==============================================================================
%% ABSTRACT
\begin{abstract}
    %Every abstract follows a similar pattern. Motivate; set aims; describe work; explain results.
    %\vskip 0.5em
    %``XYZ is bad. This project investigated ABC to determine if it was! better. ABC used XXX and YYY to implement ZZZ. This is particularly interesting as XXX and YYY have never been used together. It was found that ABC was 20\% better than XYZ, though it caused rabies in half of subjects.''

    Four dimensional space is a mathematical concept that cannot be visualised in its entirety by humans. Several paradigms exist to visualise four dimensional objects in 3D, but a viewer will only be able to see a fraction of a 4D object at one time.
    \vskip 0.5em
    Ray marching over signed distance functions provide a quick method of rendering 3D cross sections of a wide range of primitive 4D shapes.
    3D cross sections of 4D objects provides a platform to develop extended visualisations to provide a user with more information about a 4D object. Such extensions have been explored in other literature. This paper aims to evaluate the effectiveness of a series of extensions to 3D cross sections of 4D objects; each representation providing the user with more information in order to help them interpret and manipulate the object they are handling.
    \vskip 0.5em

    TODO: Results of Experiment
    It was found that ...
    
\end{abstract}

%==============================================================================
%% ACKNOWLEDGEMENTS
\renewcommand{\abstractname}{Acknowledgements}
\begin{abstract}
    I would like to thank my supervisor, Dr. John Williamson for the constant support and advice throughout the project, within and outside of the weekly meetings. Thanks to his guidance and support, I was able to take this project as far as I have, and fully explore the complexities of the fourth dimension.
\end{abstract}
%==============================================================================

% EDUCATION REUSE CONSENT FORM
% If you consent to your project being shown to future students for educational purposes then insert your name and the date below to  sign the education use form that appears in the front of the document. You must explicitly give consent if you wish to do so. If you sign, your project may be included in the Hall of Fame if it scores particularly highly.

% Please note that you are under no obligation to sign  this declaration, but doing so would help future students.

\def\consentname {Dominic Joe Subbiani} % your full name
\def\consentdate {24 January 2022} % the date you agree

\educationalconsent


%==============================================================================
\tableofcontents

%==============================================================================
%% Notes on formatting
%==============================================================================
% The first page, abstract and table of contents are numbered using Roman numerals and are not included in the page count. 

% From now on pages are numbered using Arabic numerals. Therefore, immediately after the first call to \chapter we need the call \pagenumbering{arabic} and this should be called once only in the document. 

% Do not alter the bibliography style.

% The first Chapter should then be on page 1. You are allowed 40 pages for a 40 credit project and 30 pages for a 20 credit report. This includes everything numbered in Arabic numerals (excluding front matter) up to but excluding the appendices and bibliography.

% You must not alter text size (it is currently 10pt) or alter margins or spacing.

%==================================================================================================================================

% IMPORTANT
% The chapter headings here are **suggestions**. You don't have to follow this model if it doesn't fit your project. Every project should have an introduction and conclusion, however. 

%==================================================================================================================================
\chapter{Introduction}

% reset page numbering. Don't remove this!
\pagenumbering{arabic} 

\section{The Fourth Dimension}

The world is described as three dimensional Euclidean space, conventionally referred to as $\mathbb{R}^3$. Three axes, conventionally labeled \(x\), \(y\) and \(z\), define the three degrees of translational freedom an entity can move within. All three axes that make up Euclidean space exist perpendicular to each other. Four dimensional space, conventionally referred to as $\mathbb{R}^4$, is the mathematical concept where a fourth perpendicular axes, conventionally labeled \(w\), exists perpendicular to all of the other three dimensional axes. It is impossible to visualise four dimensions in its entirety and often has to be abstracted to an analogues 3D - 2D example to reason and understand the behaviour of four dimensional entities. 

\section{Opportunities for Exploration}

Handling four dimensional space presents a number of opportunities and challenges that come with the desire to explore, interpret and manipulate higher dimensional objects. The most immediate challenge is that of being able to manipulate a four dimensional object. Rotating an object is a non trivial challenge and the most common method of rotation in 3D cannot be extended to higher dimensions. There are several ways in which users can interact with 3D objects. Another non-trivial challenge is finding intuitive ways in which a user can manipulate a 4D object through a two dimensional user interface.
\vskip 0.5em
To understand the orientation of an object, a user needs to be able to differentiate similar faces of that object from each other. Colour, patterns and textures can be applied to an object in order to assist in comprehending the current status of an object in comparison to an otherwise visually similar counterpart. An example of this is a cube rotated 90 degrees (\(\frac{\pi}{2}\) radians) and the same cube rotated by 270 degrees (\(\frac{3\pi}{2}\) radians).
\vskip 0.5em
In this paper several methods will be applied in an attempt to provide more information about a four dimensional object. Such methods include: viewing the object from other angles, attempts to visualise the object in its entirety, and abstractions of the rotation of an object in order to help understand higher dimensional rotations.

\section{Motivation}

\textbf{Motivate} first, then state the general problem clearly. 

Three dimensional space, on the surface, is very neat. There are three degrees of translational freedom expressed by the three axes \(x\), \(y\) and \(z\). There are three degrees of rotation that are most commonly expressed as rotations about each axis. 
%
This supposed tidiness has resulted in misconceptions that are rooted in to society such that nearly every digital 3D system is built upon a mathematically impure foundation. 
%
In practice the misconceptions do not matter, and the mathematics is still sound. However it is interesting nonetheless that

comes with a series of interesting problems 

what is the benefit of trying to teach people about it?

TODO: Discuss the applications of 4d
\citep{zhou_visualization_1991}

\section{Aims}

% wide range of 4 Dimensional shape. variety of ways to display. As such the usability of each representation will be experimentally validated in order to find the most intuitive and effective way to represent higher dimensional shapes. The accuracy of which the user can identify these properties and act accordingly will be measured against the representation being shown to them.

As per the motivation for this project; an effective tutorial must be compiled together in order to explain the foundations of four dimensional geometry. A system to view and interact with 4D objects must be developed and have the capability to showcase a wide variety of 4D shapes. Research into intuitive methods of user interaction should be explored, and a system to rotate an object without gimbal lock should be employed.
\vskip 0.5em
An evaluation will be conducting into what aspects of a shape are most important when trying to interpret higher dimensional geometry, using a series of tests that evaluate the understanding of the geometry, how its cross section changes under rotation, as well as a users ability to manipulate the shape.

%==================================================================================================================================
\chapter{Background}

\section{Representations of the Fourth Dimension}

There are many ways to represent the fourth dimension. Arguably the most intuitive understanding is that of a 3D cross section.

stereo graphic projection

light from center of ball

\subsection{Extensions to the 3D Cross Section}

\citep{kageyama_visualization_2015}

\citep{matsumoto_polyvision_2019}

\section{Building a 4D Object}

\subsection{Mesh Based Rendering}

\citep{bosch_n-dimensional_2020}
\citep{tianli_4d_2018}

\subsection{Ray Marching Shaders}

\citep{quilez_distance_nodate}
\citep{the_art_of_code_ray_2018}

\section{Rotating a 4D Object}

In $\mathbb{R}^3$ there are three degrees of rotational freedom. As there are three axes, the method of rotating an object has commonly been considered as rotating about an axis. However, in $\mathbb{R}^4$ there are six degrees of rotational freedom, despite there only being four axes. Instead, rotation must be considered as a rotation about a plane formed by two axes. Therefore the six planes of rotation in $\mathbb{R}^4$ would be \(xy\), \(xz\), \(yz\), \(xw\), \(yw\) and \(zw\).

\subsection{The Problems with Rotation}

Rotating an object in $\mathbb{R}^2$ is somewhat trivial. Given a vector $V_{xy}$ rotating about an origin point within the \(xy\) plane, the vector will follow a circular path dictated by the following 2D rotation matrix:
% 2D Rotation Matrix
$$V_{xy}' = V_{xy}\begin{bmatrix}
  cos \theta & - sin \theta \\
  sin \theta & cos \theta
\end{bmatrix}$$
%
Rotating an object in a space greater than two dimensions immediately becomes a non-trivial problem due to it being non-commutative. In dimensions greater than $\mathbb{R}^2$, rotation matrices can be composed into a homogeneous rotation matrix. Multiplying a vector with a such a matrix will perform a rotation but will likely encounter gimbal lock. Gimbal lock occurs when rotating an object considers each plane of rotation as independent from one another. As a result, if a particular plane becomes parallel with another, a degree of rotation is lost and the vector being rotated will not change as expected.
\vskip 0.5em
Introducing the quaternion: The quaternion is an extension of the complex number system. A quaternion has 4 components: \(x\), \(y\) and \(z\) which describe the axis of rotation, and \(w\) which describes the amount of rotation. Quaternions do not suffer from gimbal lock and can be applied to each other to perform several rotations in series.
\vskip 0.5em
Quaternions have a problem: They consider rotations as a rotation about an axis. As discussed by \citet{bosch_lets_nodate}, axial rotations are not appropriate as a generalised method of rotation across dimensions. For example, you would not consider a 2D object rotating about the \(z\) axis. It makes far more sense to stay within $\mathbb{R}^2$ and rotate about the \(xy\) plane.

\subsection{The Rotor}

Geometric Algebra, often referred to as Clifford Algebra, is a field of mathematics describing vector space. Vector space is populated by multivectors, graded by their vector components. Most notably, multivectors are defined by their associative, distributive and anti-commutative properties \citep{baker_maths_nodate}.
A grade zero multivector is a scalar number. a grade one multivector is just a vector. Grade two multivectors are known as bivectors, which build the foundation for the rotor. A trivector is a grade 3 multivector forming a parallelepiped from 3 vectors. Finally in $\mathbb{R}^4$ there is what is known as a pseudo-scalar.
\vskip 0.5em
A bivector $b_{ab}$ is made up of two vectors \(a\) and \(b\), and has, alongside its orientation in space defined by it's vector components, two important properties: The direction of rotation, from \(a\) to \(b\), and an area, dictated by the parallelogram formed by its two vector components as illustrated by \citet{slehar_clifford_2014}. A bivector in the reverse direction, i.e from \(b\) to \(a\) fulfills the anti-commutative property such that $b_{ba} = -b_{ab}$.
\vskip 0.5em
As mentioned above, a rotation should be considered as a rotation about a plane. A bivector defines the plane for a vector to rotate about. As demonstrated by \citet{mathoma_geometric_2017}, a rotor can rotate a vector following the principle that a double reflection forms a rotation \citep{mathoma_geometric_2016-2}.
\vskip 0.5em
In a given vector space greater than $\mathbb{R}^2$, a multivector greater than grade zero can be projected onto the vector space basis blades. A basis blade is a the plane formed parallel to two perpendicular axes. The \(xy\) blade is often referred to as $e_{xy}$ or $e_{01}$. Therefore a bivector in $\mathbb{R}^3$ can be decomposed into three projections onto the $e_{xy}$, $e_{xz}$ and $e_{yz}$ planes; and a bivector in $\mathbb{R}^4$ can be decomposed into 6 elements.
\vskip 0.5em
TODO: Geometric Product to rotate an object

\citep{bosch_4d_2011}

TODO: Similarities between Quaternion and Rotor
\citep{bosch_code_nodate}

%clifford algebra

%\citep{mathoma_geometric_2017}

\section{Interaction and Direct Manipulation}

\subsection{Methods of Interaction in 3D}

local, grab ball driven

global, swipe gestures (often multi-touch)

\citep{shoemake_arcball_1994}

\subsection{Methods of Interaction in 4D}

\citep{murata_interactive_2000}
\citep{kageyama_keyboard-based_2005}

\section{Teaching and Evaluation}

\citep{safrankova_van_2012}

%==================================================================================================================================
\chapter{Analysis and Requirements}

self defined project with the goal of teaching people about abstract mathematical concepts

\section{Problem Specification}

teach people, and develop something suited to teaching people

\section{Limitations}

time to develop

time per user to experiment\\
 - quantity of quality

\section{Prioritisation Using MoSCoW}

explanation of MoSCoW

\subsection*{Must Have}

be able to render multiple 4D shapes

multiple extensions to a 3d cross section of 4d space

create a tutorial video to teach and explain 4d
 - must cover the geometry
 - must cover the rotation

\subsection*{Should Have}

be able to interact with them in 4D directly

multiple tests to evaluate a persons understanding

\subsection*{Could Have}

multiple methods of direct interaction/manipulation

%===

What is the problem that you want to solve, and how did you arrive at it?

%==================================================================================================================================
\chapter{Design}

\section{Tutorial}

requirements for teaching

 - explain 4D space, why it cannot be visualised
 - explain 4D rotation in relation to 2D and 3D rotation

\section{Experiment}

requirements for testing

 - be able to understand the shapes
 - be able to interpret the rotation of a shapes
 - be able to manipulate a shape

%==================================================================================================================================
\chapter{Implementation}

\section{Building 4D Objects With Ray Marching}

shapes defined with signed distance functions

relative to the camera - slice is based on W coordinate of camera based

algorithm

\subsection{Derivation of 4D Shapes}


\section{Extending 3 Dimensional Cross Sections - Representing 4D Objects}

timeline: simple - ellyptical version
 - ellypitcal adds more information, but not as intuitive as straight line\\
onionskin - similar to timeline, but not very intuitive for users unfamiliar with 4D\\
multi-view: polyvision\\
abstraction of 4D rotation using 3D rotation - does not extend well to n dimensions unlike all other propersitions

\section{Rotation of a 4D Vector}

initial manual derivation

sympy and galgebra

equations

\section{Methods of Rotation}

\subsection{Swipe Based Input - Rotation About The Global Axes}

intuitive\\
does not define "how rotated" 4D axes are relative to the local coordinates of the shape

\subsection{4D Grab Ball - Rotation About The Local Axes}

idea\\
progress\\
why it failed

\section{Texturing Ray Marched Objects}

normal based projection

\section{Building the Experiment}

\subsection{}

\subsection{Data Collection}

SimpleJSON

Email - challenges\\
Copy \& Paste - challenges

%==================================================================================================================================
\chapter{Evaluation} 

\section{Aims} 

\section{Experimental Design}

\subsection{Preliminary Research}

what is the best way to explain concepts\\
try teach some friends the basics and see what explanations work best

the best ways to teach people about geometry\\
 - reference that paper about 5 stages or something

between users experiment
 - test user with every representation

\subsection{Repeated Preliminary Experiments}

build system\\
test it\\
improvements\\

Need to provide clear explanations of concepts early on

provide time to play with object before tests to learn controls and behaviour

\subsection{Tasks and Parameters}

shape match\\
 - all shapes\\
 - diffuse texture - patterns give it away\\
rotation match\\
 - not diffuse texture and no sphere - no surface imperfections and can be impossible to tell if certain rotations occuring\\
 - 1-2 4D rotations and occasionally 3d rotations\\
pose match\\
 - randomly oriented match shape

random order of representations

\subsection{Limitations}

\section{Results}


%==================================================================================================================================
\chapter{Conclusion}    
Summarise the whole project for a lazy reader who didn't read the rest (e.g. a prize-awarding committee).
\section{Guidance}
\begin{itemize}
    \item
        Summarise briefly and fairly.
    \item
        You should be addressing the general problem you introduced in the
        Introduction.        
    \item
        Include summary of concrete results (``the new compiler ran 2x
        faster'')
    \item
        Indicate what future work could be done, but remember: \textbf{you
        won't get credit for things you haven't done}.
\end{itemize}

%==================================================================================================================================
%
% 
%==================================================================================================================================
%  APPENDICES  

\begin{appendices}

\chapter{Appendices}

Typical inclusions in the appendices are:

\begin{itemize}
\item
  Copies of ethics approvals (required if obtained)
\item
  Copies of questionnaires etc. used to gather data from subjects.
\item
  Extensive tables or figures that are too bulky to fit in the main body of
  the report, particularly ones that are repetitive and summarised in the body.

\item Outline of the source code (e.g. directory structure), or other architecture documentation like class diagrams.

\item User manuals, and any guides to starting/running the software.

\end{itemize}

\textbf{Don't include your source code in the appendices}. It will be
submitted separately.

\end{appendices}

%==================================================================================================================================
%   BIBLIOGRAPHY   

% The bibliography style is abbrvnat
% The bibliography always appears last, after the appendices.

\bibliographystyle{abbrvnat}

\bibliography{l4proj}

\end{document}
