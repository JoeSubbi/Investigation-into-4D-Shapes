% REMEMBER: You must not plagiarise anything in your report. Be extremely careful.

\documentclass{l4proj}

    
%
% put any additional packages here
%

\begin{document}

%==============================================================================
%% METADATA
\title{Level 4 Project: An Investigation of How Best to Represent a Cross Section of a 4 Dimensional Objects}
\author{Joe Subbiani}
\date{March 18, 2022}

\maketitle

%==============================================================================
%% ABSTRACT
\begin{abstract}
    %Every abstract follows a similar pattern. Motivate; set aims; describe work; explain results.
    %\vskip 0.5em
    %``XYZ is bad. This project investigated ABC to determine if it was! better. ABC used XXX and YYY to implement ZZZ. This is particularly interesting as XXX and YYY have never been used together. It was found that ABC was 20\% better than XYZ, though it caused rabies in half of subjects.''
    4 Dimensional space is a mathematical concept that cannot be visualised in its entirety by humans. Several paradigms exist to visualise 4 Dimensional objects in 3D, but a viewer will only be able to see a fraction of a 4D object at one time.
    \vskip 0.5em
    Ray marching over signed distance functions were used in order to render 3D cross sections of a wide range of simple 4D shapes.
    Using 3D cross sections of 4D objects provides a platform to develop extended visualisations to provide a user with more information about a 4D object. Such extensions have been explored in other literature. This paper aims to evaluate the effectiveness of a series of ways to representing 3D cross sections of 4D objects; each representation providing the user with extra information in order to help them interpret and manipulate the object they are handling.
    
\end{abstract}

%==============================================================================
%% ACKNOWLEDGEMENTS
\renewcommand{\abstractname}{Acknowledgements}
\begin{abstract}
    I would like to thank my supervisor, Dr. John Williamson for the constant support and advice throughout the project, within and outside of the weekly meetings. Thanks to his guidance and support, I was able to take this project as far as I have, and fully explore the complexities of the fourth dimension.
\end{abstract}
%==============================================================================

% EDUCATION REUSE CONSENT FORM
% If you consent to your project being shown to future students for educational purposes then insert your name and the date below to  sign the education use form that appears in the front of the document. You must explicitly give consent if you wish to do so. If you sign, your project may be included in the Hall of Fame if it scores particularly highly.

% Please note that you are under no obligation to sign  this declaration, but doing so would help future students.

\def\consentname {Dominic Joe Subbiani} % your full name
\def\consentdate {24 January 2022} % the date you agree

\educationalconsent


%==============================================================================
\tableofcontents

%==============================================================================
%% Notes on formatting
%==============================================================================
% The first page, abstract and table of contents are numbered using Roman numerals and are not included in the page count. 

% From now on pages are numbered using Arabic numerals. Therefore, immediately after the first call to \chapter we need the call \pagenumbering{arabic} and this should be called once only in the document. 

% Do not alter the bibliography style.

% The first Chapter should then be on page 1. You are allowed 40 pages for a 40 credit project and 30 pages for a 20 credit report. This includes everything numbered in Arabic numerals (excluding front matter) up to but excluding the appendices and bibliography.

% You must not alter text size (it is currently 10pt) or alter margins or spacing.

%==================================================================================================================================

% IMPORTANT
% The chapter headings here are **suggestions**. You don't have to follow this model if it doesn't fit your project. Every project should have an introduction and conclusion, however. 

%==================================================================================================================================
\chapter{Introduction}

% reset page numbering. Don't remove this!
\pagenumbering{arabic} 

\section{The Fourth Dimension}

\textbf{Motivate} first, then state the general problem clearly. 

math concept  

cannot see it  

\section{Opportunities for Exploration}

texturing objects

methods of rotating objects

viewing objects from different perspectives

attempting to abstract the rotation of an object to something that can be visualised a bit clearer

trying to better visualise the object in its entirety

\section{Motivation}

comes with a series of interesting problems

what is the benefit of trying to teach people about it?

\section{Aims}

develop and test a series of extensions to taking a cross section of a 4D object and seeing what is more intuitive/easier to use

%==================================================================================================================================
\chapter{Background}

\section{Representations of the Fourth Dimension}

stereo graphic projection

light from center of ball

cross sections

\section{Building a 4D Object}

\subsection{Mesh Based Rendering}

\subsection{Ray Marching Shaders}

\section{Rotating a 4D Object}

\subsection{The Problem}

quaternions leave dimension

\subsection{The Rotor}

geometric algebra

\section{Interaction and Direct Manipulation}

\subsection{Methods of Interaction in 3D}

\subsection{Methods of Interaction in 4D}

%==================================================================================================================================
\chapter{Analysis and Requirements}

self defined project with the goal of teaching people about abstract mathematical concepts

\section{Problem Specification}

teach people, and develop something suited to teaching people

\section{Limitations}

time to develop

time per user to experiment\\
 - quantity of quality

\section{Prioritisation Using MoSCoW}

explanation of MoSCoW

\subsection*{Must Have}

be able to render multiple 4D shapes

multiple extensions to a 3d cross section of 4d space

create a tutorial video to teach and explain 4d
 - must cover the geometry
 - must cover the rotation

\subsection*{Should Have}

be able to interact with them in 4D directly

multiple tests to evaluate a persons understanding

\subsection*{Could Have}

multiple methods of direct interaction/manipulation

%===

What is the problem that you want to solve, and how did you arrive at it?

%==================================================================================================================================
\chapter{Design}

\section{Tutorial}

requirements for teaching

 - explain 4D space, why it cannot be visualised
 - explain 4D rotation in relation to 2D and 3D rotation

\section{Experiment}

requirements for testing

 - be able to understand the shapes
 - be able to interpret the rotation of a shapes
 - be able to manipulate a shape

%==================================================================================================================================
\chapter{Implementation}

\section{Building 4D Objects With Ray Marching}

shapes defined with signed distance functions

relative to the camera - slice is based on W coordinate of camera based

algorithm

\subsection{Derivation of 4D Shapes}


\section{Extending 3 Dimensional Cross Sections - Representing 4D Objects}

timeline: simple - ellyptical version
 - ellypitcal adds more information, but not as intuitive as straight line\\
onionskin - similar to timeline, but not very intuitive for users unfamiliar with 4D\\
multi-view: polyvision\\
abstraction of 4D rotation using 3D rotation - does not extend well to n dimensions unlike all other propersitions

\section{Rotation of a 4D Vector}

initial manual derivation

sympy and galgebra

equations

\section{Methods of Rotation}

\subsection{Swipe Based Input - Rotation About The Global Axes}

intuitive\\
does not define "how rotated" 4D axes are relative to the local coordinates of the shape

\subsection{4D Grab Ball - Rotation About The Local Axes}

idea\\
progress\\
why it failed

\section{Texturing Ray Marched Objects}

normal based projection

\section{Building the Experiment}

\subsection{}

\subsection{Data Collection}

SimpleJSON

Email - challenges\\
Copy \& Paste - challenges

%==================================================================================================================================
\chapter{Evaluation} 

\section{Aims} 

\section{Experimental Design}

\subsection{Preliminary Research}

what is the best way to explain concepts\\
try teach some friends the basics and see what explanations work best

the best ways to teach people about geometry\\
 - reference that paper about 5 stages or something

between users experiment
 - test user with every representation

\subsection{Repeated Preliminary Experiments}

build system\\
test it\\
improvements\\

Need to provide clear explanations of concepts early on

provide time to play with object before tests to learn controls and behaviour

\subsection{Tasks and Parameters}

shape match\\
 - all shapes\\
 - diffuse texture - patterns give it away\\
rotation match\\
 - not diffuse texture and no sphere - no surface imperfections and can be impossible to tell if certain rotations occuring\\
 - 1-2 4D rotations and occasionally 3d rotations\\
pose match\\
 - randomly oriented match shape

random order of representations

\subsection{Limitations}

\section{Results}


%==================================================================================================================================
\chapter{Conclusion}    
Summarise the whole project for a lazy reader who didn't read the rest (e.g. a prize-awarding committee).
\section{Guidance}
\begin{itemize}
    \item
        Summarise briefly and fairly.
    \item
        You should be addressing the general problem you introduced in the
        Introduction.        
    \item
        Include summary of concrete results (``the new compiler ran 2x
        faster'')
    \item
        Indicate what future work could be done, but remember: \textbf{you
        won't get credit for things you haven't done}.
\end{itemize}

%==================================================================================================================================
%
% 
%==================================================================================================================================
%  APPENDICES  

\begin{appendices}

\chapter{Appendices}

Typical inclusions in the appendices are:

\begin{itemize}
\item
  Copies of ethics approvals (required if obtained)
\item
  Copies of questionnaires etc. used to gather data from subjects.
\item
  Extensive tables or figures that are too bulky to fit in the main body of
  the report, particularly ones that are repetitive and summarised in the body.

\item Outline of the source code (e.g. directory structure), or other architecture documentation like class diagrams.

\item User manuals, and any guides to starting/running the software.

\end{itemize}

\textbf{Don't include your source code in the appendices}. It will be
submitted separately.

\end{appendices}

%==================================================================================================================================
%   BIBLIOGRAPHY   

% The bibliography style is abbrvnat
% The bibliography always appears last, after the appendices.

\bibliographystyle{abbrvnat}

\bibliography{l4proj}

\end{document}
